%~ \documentclass[12pt,twoside,a4paper,titlepage]{scrbook}
\documentclass{scrbook}

%~ Anhang ([toc,page] würde noch eine extra Seite for dem Anhang anfügen)
\usepackage[toc]{appendix}

%~ Mathematische Symbole
\usepackage{amssymb, amsmath, amscd, amsthm}

%~ Sprache
\usepackage[ngerman]{babel}

% named cross references
\usepackage{nameref}

\usepackage[T1]{fontenc}
\usepackage[utf8]{inputenc}

%~ Abbildungen (wichtig für die pdf_tex-Dateien)
\usepackage{graphicx}
\usepackage{color}
%~ \graphicspath{{figures/}} <-- geht nicht.
\usepackage{import}
\usepackage{wrapfig}

%~ fancyhdr – Extensive control of page headers and footers in LaTeX2ε
%~ \usepackage{fancyhdr}

\newcommand{\includegraphicsmaybe}[1]{\IfFileExists{#1}{\includegraphics[width=\textwidth]{#1}}{\includegraphics[width=\textwidth]{abbildungen/platzhalter.jpg}}}

%~ Blindtext weil ich jetzt kein ganzes Skript schreiben möchte - liest eh keiner
\usepackage{blindtext}

%~ Bequeme Befehle wie \de­gree, \cel­sius, \pert­hou­sand, \mi­cro und \ohm
\usepackage{gensymb} 

\begin{document}

\begin{titlepage}
	\begin{center}
		\vspace*{5em}
		{\huge Chemie und so}\\[1.5em]
		{\Large Wintersemester 2013}\\[4em]	
		{\large \textsc{Vorlesungsskript}}\\[4em]
		{\large Fakult"at f"ur Mathematik, Informatik und Naturwissenschaften \\
		der Universit"at Hamburger}\\[1.5em]
	\end{center}
\end{titlepage}

%~ Einführung
\chapter{Einführung}
%~ Einführung
%~
%~
\Blindtext[3][3]

\begin{figure}
	\centering
	\includegraphicsmaybe{abbildungen/Argon_Sputteryield_Laegreid.jpg}
	\caption{\blindtext}
\end{figure}

\begin{figure}
	\centering
	\includegraphicsmaybe{geschuetztes_abbildungen/BJJ_04_Si_03_skala.jpg}
	\caption{\blindtext \textbf{Diese Abbildung ist urheberrechtlich geschützt.}}
\end{figure}


\blinddocument


%~ Alkaloide: Hygrin, Solanin usw.
\chapter{Alkaloide}
\input{kapitel/alkoloide.tex}

%~ Halogene: fluorid, chloride etc.
\chapter{Halogene}
\input{kapitel/halogene.tex}

%~ Ausblick auf die nächste Vorlesung
\chapter{Ausblick}
\input{kapitel/ausblick.tex}


\end{document}
